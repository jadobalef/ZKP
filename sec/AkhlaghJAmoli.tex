\section{اهمیت علم اخلاق در لسان آیت الله العظمی جوادی آملی}
آیت الله العظمی جوادی آملی معتقدند علم اخلاق از علم فقه و اصول و امثالهم دارای اهمیت و دشواری بیشتری می‌باشد چرا که اگر کسی بخواهد متخلق شود، باید مظهر «هو الخالق» باشد و این کار آسانی نیست. 

معظم له بین علم اخلاق با موعظه تفاوت قائل هستند و بیان می دارند:
موعظه غير از علم اخلاق است؛ موعظه علم نيست، يک سفارشات کلّي است؛ آدم باتقوا باشيد آدم خوبي باشيد منزّه باشيد، اينها سفارشات توده مردم است؛ اما اخلاق، علم است موضوع دارد محمول دارد دليل دارد نتيجه دارد و جزء علوم بسيار دشوار و پيچيده است و اگر کسي بخواهد متخلّق بشود بايد مظهر «هو الخالق» باشد که خود را بسازد.
 اخلاق هم جوهره او با موعظه فرق دارد هم رهآوردش با موعظه فرق دارد. بنابراين اين از فقه و اصول اگر مشکلتر و علميتر نباشد، آسان تر نيست،  روح و تجرّد روح را ثابت کردن، شئون روح را ثابت کردن، سود و زيان روح را ثابت کردن، ابدي بودن روح را ثابت کردن، توانمندي روح را ثابت کردن، مظهر «هو الخالق» بودنِ روح را ثابت کردن، کار آساني نيست.

\subsection{توصيه اهل بيت(عليهم السّلام) به ضرورت جهاد با هواي نفس}
آیت الله العظمی جوادی آملی بر ضرورت جهاد با هوای نفس در لسان اهل بیت علیهم السلام اشاره دارند و اظهار می‌دارند: یکی از از روایاتی که ائمه(عليهم السلام) می‌فرمایند این است که: «جَاهِدُوا أَهْوَاءَكُمْ كَمَا تُجَاهِدُونَ أَعْدَاءَكُم‏»؛ همان طوري که شما با دشمن بيروني ميجنگيد، بايد با دشمن دروني که هوا و هوس است بجنگيد.

\subsection{اشک و آه، بهترین سلاح مبارزه با هوای نفس}
معظم له  با بیان اینکه طبق روایات اهل بیت، اشك و آه، بهترين سلاح مبارزه با هواي نفس است، اظهار می دارند:
اکبر بودن و اصغر بودنِ جهاد، به اکبر بودن و اصغر بودنِ عداوت و مبارزه و کيفيت مبارزه است. چون ضرر دشمن درون سنگينتر از دشمن بيرون است و مبارزه با او هم دشوارتر از مبارزه با بيرون است؛ لذا فرمود از جهاد اصغر آمديد وارد جهاد اکبر ميشويد. تهيه سلاح و با آهن جنگيدن کار هر کسي است، چه اينکه شما ميبينيد الآن چه شرق چه غرب برخي از کارخانههاي توليد سلاح آدمکشي، اينها دو سه شيفته شبانهروز دارند کار ميکنند. آهنسازي کار آساني است؛ اما طبق بيان نوراني حضرت امير که در دعاي «کميل» ميخوانيد فرمود: در مبارزه با نفس «وَ سِلَاحُهُ‏ الْبُكَاء»[1]آهنساختن آسان است! اما  آه پيدا کردن کار هر کسي نيست، آهي ميخواهد، يک جگر سوختهاي ميخواهد، يک دل مطمئنّي ميخواهد که انسان بتواند اين آز و اين هوا را رام بکند. لذا فرمود: «أَعْدَي عَدُوِّكَ نَفْسُكَ الَّتِي بَيْنَ جَنْبَيْكَ»
بنابراين طبق «جَاهِدُوا أَهْوَاءَكُمْ كَمَا تُجَاهِدُونَ أَعْدَاءَكُم‏» که يک اصل است، طبق «أَعْدَي عَدُوِّكَ نَفْسُكَ الَّتِي بَيْنَ جَنْبَيْكَ» که اصلي ديگر است، ما با دشمني بزرگ و بزرگتر يا دشمن کوچک و بزرگ درگير هستيم و اگر دشمن بزرگ يا بزرگتر را کوبيديم بر دشمن بيروني مسلط خواهيم شد.

ایشان ادامه می دهند:
در جريان جهاد با بيگانه انسان سه حالت دارد:
يا پيروز ميشود،يا شهيد ميشود، يا اسير.
در جهاد درون هم اين‎چنين است؛ يعني کسي بخواهد با نفس، با غرور، با خواستههاي نفس که او را به طرف گناه ميکشانند، مبارزه کند، اين سه حالت دارد:
يا انسان در جهاد با هواي نفس شهيد ميشود، يا فاتح ميشود، يا اسير.
در جهاد اصغر اگر اسير شد آن دشمن حداکثر مال و جان را بگيرد کاري با دين ندارد؛ ولي در جهاد اوسط، جهاد درون، جهاد با هوای نفس اگر کسي اسير شد کلّ دين را باخته است و اگر شهيد شد، شهيد در جهاد اوسط کسي است که چهار تا تير ميزند چهار تا تير ميخورد، اگر چهار تا نگاه حرام کرد، چهار تا توبه هم ميکند، چهار تا کار خير هم ميکند، چهار تا اطاعت ميکند، دو تا تير ميزند دو تا تير ميخورد تا سرانجام دينش را حفظ ميکند و ميميرد. اينکه در روايات ما دارد که اگر کسي محبّ اهل بيت(عليهم السلام) بود اين ولو در بستر بيماري بميرد «مَاتَ شَهِيداً»[2] براي همين است. اين شهيد در جهاد اصغر نيست، اين شهيد در جهاد اوسط است؛ براي اينکه چندين تير زد چندين تير خورد، بالاخره تسليم نشد، دينش را حفظ کرد، عقيدهاش را حفظ کرد، احترام به قرآن را حفظ کرد، احترام به عترت را حفظ کرد، شيعه خاص بود و رحلت کرد.

\subsection{جنگ بين عقل و قلب محور مبارزه در جهاد اکبر}
معظم له محور مبارزه در جهاد اکبر را جنگ بین عقل و قلب بیان داشته و اظهار می دارند:
در جهاد اوسط که مصطلح جهاد اکبر است، جنگ بين عقل است و نفس، بين علم است و جهل؛ عقل ميخواهد اطاعت کند نفس ميخواهد معصيت کند ، اين در حقيقت جهاد اوسط است نه جهاد اکبر، چون جهاد اکبر مطرح نيست در دسترس نيست برای اوحدي اهل سير و سلوک است اين وسطي را ميگويند جهاد اکبر برابر آن حديث معروف. جهاد اکبر آن است که بين قلب و عقل است، سخن از نفس نيست، سخن از معصيت و ترک اطاعت و اينها نيست؛ عقل ميگويد من ميخواهم بفهمم، قلب ميگويد فهميدن کافي نيست بايد ديد. شما دليل اقامه ميکنيد دليل قرآني، روايي و عقلي که بهشت هست جهنم هست قيامت هست اينها هنر نيست، هنر اين است که بله جهنّم هست بهشت هست آدم بهشت را ببيند جهنّم را ببيند، قلب در جهاد اکبر به عقل ميگويد همراه من بيا تا بهشت را ببيني! بهشت الآن موجود است جهنّم موجود است، اين را بايد ديد، اين است که ديدن سخت است بايد فهميد.

\subsection{تبيين  اهمیت اخلاق و علم اخلاق}
آیت الله العظمی جوادی آملی سپس در تبيين  اهمیت اخلاق و علم اخلاق ، بیان می دارند:
ما يک جهاد اصغر داريم که جنگ با بيرون است که بيگانه بر مرز و بوم اين کشور دخالت نکند، نفوذ پيدا نکند و مانند آن. يکي اينکه نفس و هواي نفس نفوذ پيدا نکند. يکي اينکه به مفهوم بسنده نکنيم به مشهود برسيم، به برهان حصولي بسنده نکنيم به برهان شهودي راه بيابيم که آن اثر فراواني دارد.
حالا اگر کسي در ميدان جهاد اصغر شهيد شد، آن همه برکات را قرآن کريم براي او مشخص کرد؛ هم فرمود که  درباره شهيد نگوييد مردهاند! اينها زندهاند برکات فراواني دارند، و بشارتهایی که در قرآن برای شهدا ذکر شده است.
اما اگر شهيد جبهه جهاد اوسط بود، او چه نشاطي دارد؟ او چه رزقي دارد؟ او چگونه از مقام «عنداللّهي» برخوردار است؟ اگر شهيد در جهاد اصغر ﴿عِنْدَ رَبِّهِمْ يُرْزَقُونَ﴾ شد، شهيد در جهاد اوسط چه خواهد بود؟ چه رسد به شهيد در جهاد اکبر. آن کسي که آن قدر چشم و گوش او پاک است که در صدد اين است که بهشت را ببيند جهنّم را ببيند اين چقدر چشم پاک ميخواهد؟! اين چقدر گوش پاک ميخواهد!؟ اين چقدر زبان پاک ميخواهد!؟ يک بيان نوراني از پيغمبر(صلي الله عليه و آله و سلم) رسيد فرمود: «طَهِّرُوا أَفْوَاهَكُمْ‏»؛ یعنی دهانتان را پاک کنيد، نه دندان خود را مسواک کنيد؛ مسواک کردنِ دندان يک سنّت فقهي است ثواب هم دارد آثار پزشکي و بهداشتي هم دارد؛ اما حضرت در اين حديث نوراني فرمود: دهن را پاک کنيد، نه أسنان و دندان را! «طَهِّرُوا أَفْوَاهَكُمْ‏»؛ اين دهن را پاک کنيد، چرا؟ «فَإِنَّهَا طُرُقُ الْقُرْآن‏‏»؛چون قرآن ميخواهد از اين دهن عبور کند. فرمود قرآن بخوانيد، بخواهيد نماز بخوانيد که حمد و سورهاش قرآن است. مگر نبايد از اين دهن بيرون بيايد؟! اين فضا را پاک کنيد، حرف بد از اين دهن بيرون نيايد، غذاي شبههناک به اين دهن وارد نشود، دهن را پاک کنيد نه دندان را و اين جانکندن ميخواهد.

 اگر کسي در فضاي جهاد اکبر، نه جهاد اوسط، شربت شهادت نوشيد اين چه فرح و نشاطي دارد؟ اين چه روزياي دارد؟ اين چه استبشاري دارد؟ هم روزي او، هم فرح او، هم استبشار او اکبر از شهيد در جهاد اوسط است که آنچه براي جهاد اوسط است، اوسع و بالاتر از شهيد در جهاد اصغر هست. اين وظيفه ماست که اخلاق اينها را نشان ميدهد.
 حالا روشن شد که علم اخلاق به مراتب از فقه دشوارتر است به مراتب از اصول دشوارتر است، چون از سنخ بناي عقلا و فهم و لغت نيست که انسان با اينها مشکل خود را حلّ کند! مگر اثبات وجود روح، تجرّد روح، آشنايي به ابديت روح، آشنايي به خلود روح، آشنايي به شئون گوناگون روح، اين به بناي عقلا و فهم عرف حلّ ميشود!؟ لذا مسئله اخلاق اوّلين مسئله و پيچيدهترين مسئله ماست، مائيم و ابديت ما، ميخواهيم او را ببينيم و اگر خواستيم برهان اقامه کنيم دليل خوب هست؛ ولي با برهان هرگز کام کسي شيرين نميشود. ببينيد ممکن است استادي در کلاسي زنبور عسل را کندوي عسل را، گلهاي عسلي را ، خود عسل را براي دانشجويان تشريح کند آنها کاملاً بفهمند که چيست؛ اما کام آنها شيرين نميشود. با تدريس کندوداري کام کسي شيرين نميشود، با تدريس فقه و اصول و فلسفه و کلام و اخلاق کسي متخلّق بشود که بتواند هنرمندانه خود را زيبا بياورد، این حاصل نميشود. اين را ما بايد بسازيم، ساختن آن هم آسان است؛ يعني در دنيا ممکن است، اصلاً اخلاق برای همين است که انسان خَلق کند، اين خلق همان خلقت خاص است.

\subsection{پاورقی}

[1] مصباح المتهجد و سلاح المتعبد، ج‏2، ص850.

\noindent
[2] دعائم الإسلام، ج1، ص217: «عَنْ عَلِيٍّ ع أَنَّهُ قَالَ: الْمَرِيضُ فِي سِجْنِ اللَّهِ مَا لَمْ يَشْكُ إِلَي عُوَّادِهِ تُمْحَي سَيِّئَاتُهُ وَ أَيُّ مُؤْمِنٍ مَاتَ مَرِيضاً مَاتَ‏ شَهِيدا ...». نهج البلاغة (للصبحي صالح)، خطبه190، ص283: «... فَإِنَّهُ مَنْ مَاتَ مِنْكُمْ عَلَي فِرَاشِهِ وَ هُوَ عَلَي مَعْرِفَةِ حَقِّ رَبِّهِ وَ حَقِّ رَسُولِهِ وَ أَهْلِ بَيْتِهِ مَاتَ‏ شَهِيدا ...».

