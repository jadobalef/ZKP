\section{
دعا
}
دعا خیلی مهم است.
دعا مهم ترین چیز در زندگی مومنانه است.
وقتی دعا می کنیم چیزی را از کسی می خواهیم که همه چیز در دست اوست.
همه چیز در دست اوست و سایر عوامل وسیله هستند.
اول خدا دوم فلان چیز و فلان کس معنا ندارد.
اول و آخر خداست.
و بقیه چیزها وسیله هستند.
بنابراین اگر چیزی بخواهد بشود در این دنیا با دعا باید بشود.
چون فقط خواستن از او مهم است.

اما خواستن از او تنها نشستن و خواستن نیست.
خواستن از خدا با خواستن و اراده شخص پیوند خورده است.
یعنی وقتی چیزی را می خواهی از خدا یعنی داری تعهد می دهی و اراده می کنی که در آن جهت تلاش کنی.
یعنی داری هدف گذاری می کنی.
یعنی داری می گویی خدایا واقعا می خواهم.
اگر فقط به زبان و به درخواست یک طرفه باشد دعا نیست.
خدا که انسان نیست که به او چیزی را سفارش بدهی و او برای تو مهیا کند.
از خدا خواستن یعنی اراده کردن و تلاش کردن در آن جهت.
از این منظر دعا جهت دهنده در زندگی است.

به همین دلیل است که گفته می شود که ادعیه دینی ما پر از نکات اخلاقی است.
چون فقظ خواندن نیست.
ممکن است عده ای فقط بخوانند و ثواب ببرند.
ولی ثواب آن ها تنها در حفظ ادعیه است.
دعای واقعی چیز دیگری است.
دعای واقعی جهت دهنده و تعیین کننده هدف و اراده شخص مومن است.
به او می گوید چه باید بکند.

مثال: وقتی می گوییم خداوندا به ما فرزندان صالح عطا بفرما یعنی اینکه خدایا عطا بفرما و در عمل هم ما می رویم برای این منظور تلاش می کنیم (مثلا مطالعه می کنیم).
این تلاش همان بندگی هست که هدف خلقت است.
و خداوند نیز آن خواسته را به وسیله ای به ما لطف می کند.

اگر دعا کردیم و نشد چه؟
اگر هدف زندگی بندگی باشد باید بدانیم که آن چه ما از خدا می خواهیم تنها در جهت بندگی است و لاغیر.
پس اگر خواستیم و نشد دیگر از کنترل ما خارج است.
و اگر چیزی از کنترل ما خارج باشد مورد عفو خدا قرار میگیریم.
و بدان که چیزی نیست که در توان خدا نباشد.


و در پایان باید بگویم که کمتر به کفر راحتی جویی کن.
الحمدلله رب العالمین
